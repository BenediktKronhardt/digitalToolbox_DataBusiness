% Comment out the examples and define your own acronyms here following this pattern:
%
% \acro{code}[Short Name]{Long Name}
%
% The acronyms must be sorted by hand, they appear in the list of acronyms in the
% order the are listed here.
%
% In the text, then use \ac{code} everywhere in the text where you want to use
% the acronym / abbreviation. When used for the first time, the long name followed
% by the acronym in parenthesis will be inserted, after that the acronym only.
%
% Another example (from the documentation of LaTeX package 'acronym'):
%
% \acro{H2O}[$\mathrm{H_2O}$]{water}
%
% Then \acs{H2O} gets “H2O” and \acl{H2O} prints “water”.
%
% You can use \acp{code} in the text to use an automatic plural (s is added to
% automatically the short or long name). If the plural is not standard, use an
% additional line for the acronym definition, where the code is the same but
% short and long name is adjusted to the plural form:
%
% \acrodefplural{code}[Short Names]{Long Names}
%
% Other example commands that can be used in the text:
%
% \Ac{code}  % Works in the same way as \ac, but starts the long form with an
%            % upper case
%
% \acf{code} % If later in the text again the Full Name of the acronym should be
%            % printed, use the command
%
% \acp{code} % Works in the same way as \ac, but makes the short and/or long forms
%            % into plurals.
%
% \Acp{code} % Works in the same way as \acp, but starts the long form with an
%            % upper case letter.
%
% See the manual for a complete description:
%
% https://ctan.org/pkg/acronym?lang=en
%
\acro{cran}[CRAN]{Comprehensive R Archive Network}
% This example does not work while using the flextable package. If you use this
% package for displaying tables, then make sure you do not use the math environment
% inside of abbreviations:
\acro{H2O}[$\mathrm{H_2O}$]{water}
\acro{ide}[IDE]{integrated development environment}
\acrodefplural{ide}[IDEs]{integrated development environments}
\acro{glm}[GLM]{generalized linear model}
