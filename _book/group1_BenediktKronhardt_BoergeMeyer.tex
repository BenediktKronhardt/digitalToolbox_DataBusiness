%%
% Copyright (c) 2017 - 2022, Ulf Köther;
% Copyright (c) 2017 - 2022, Pascal Wagler;
% Copyright (c) 2014 - 2022, John MacFarlane
%
% All rights reserved.
%
% Redistribution and use in source and binary forms, with or without
% modification, are permitted provided that the following conditions
% are met:
%
% - Redistributions of source code must retain the above copyright
% notice, this list of conditions and the following disclaimer.
%
% - Redistributions in binary form must reproduce the above copyright
% notice, this list of conditions and the following disclaimer in the
% documentation and/or other materials provided with the distribution.
%
% - Neither the name of John MacFarlane nor the names of other
% contributors may be used to endorse or promote products derived
% from this software without specific prior written permission.
%
% THIS SOFTWARE IS PROVIDED BY THE COPYRIGHT HOLDERS AND CONTRIBUTORS
% "AS IS" AND ANY EXPRESS OR IMPLIED WARRANTIES, INCLUDING, BUT NOT
% LIMITED TO, THE IMPLIED WARRANTIES OF MERCHANTABILITY AND FITNESS
% FOR A PARTICULAR PURPOSE ARE DISCLAIMED. IN NO EVENT SHALL THE
% COPYRIGHT OWNER OR CONTRIBUTORS BE LIABLE FOR ANY DIRECT, INDIRECT,
% INCIDENTAL, SPECIAL, EXEMPLARY, OR CONSEQUENTIAL DAMAGES (INCLUDING,
% BUT NOT LIMITED TO, PROCUREMENT OF SUBSTITUTE GOODS OR SERVICES;
% LOSS OF USE, DATA, OR PROFITS; OR BUSINESS INTERRUPTION) HOWEVER
% CAUSED AND ON ANY THEORY OF LIABILITY, WHETHER IN CONTRACT, STRICT
% LIABILITY, OR TORT (INCLUDING NEGLIGENCE OR OTHERWISE) ARISING IN
% ANY WAY OUT OF THE USE OF THIS SOFTWARE, EVEN IF ADVISED OF THE
% POSSIBILITY OF SUCH DAMAGE.
%%
%
% This template is build upon the 'default.latex' template from pandoc:
%
% https://github.com/jgm/pandoc/tree/master/data/templates
%
% Version from: 2022-05-17
%
% Pandoc version: 2.18.1
%
% Additionally, some code was borrowed from the Eisvogel pandoc template
% by Pascal Wagler:
%
% https://github.com/Wandmalfarbe/pandoc-latex-template
%
%
% All changes were applied to build a report- / thesis-template for the HSBA.
%
%%

% Options for packages loaded elsewhere
\PassOptionsToPackage{unicode}{hyperref}
\PassOptionsToPackage{hyphens}{url}
%
\documentclass[
  11pt,
  a4paper,
  twoside]{scrbook}
\usepackage{amsmath,amssymb}
\usepackage[]{cmbright}

\usepackage{setspace}

\usepackage{iftex}
\ifPDFTeX
  \usepackage[T1]{fontenc}
  \usepackage[utf8]{inputenc}
  \usepackage{textcomp} % provide euro and other symbols
\else % if luatex or xetex
  \usepackage{unicode-math}
  \defaultfontfeatures{Scale=MatchLowercase}
  \defaultfontfeatures[\rmfamily]{Ligatures=TeX,Scale=1}
  \setmonofont[Scale=0.9]{Source Code Pro}
\fi
% Use upquote if available, for straight quotes in verbatim environments
\IfFileExists{upquote.sty}{\usepackage{upquote}}{}
\IfFileExists{microtype.sty}{% use microtype if available
  \usepackage[]{microtype}
  \UseMicrotypeSet[protrusion]{basicmath} % disable protrusion for tt fonts
}{}
\makeatletter
\@ifundefined{KOMAClassName}{% if non-KOMA class
  \IfFileExists{parskip.sty}{%
    \usepackage{parskip}
  }{% else
    \setlength{\parindent}{0pt}
    \setlength{\parskip}{6pt plus 2pt minus 1pt}}
}{% if KOMA class
  \KOMAoptions{parskip=half}}
\makeatother
\usepackage{xcolor}
\usepackage[top=30mm,bottom=30mm,left=25mm,right=25mm]{geometry}
\usepackage{color}
\usepackage{fancyvrb}
\newcommand{\VerbBar}{|}
\newcommand{\VERB}{\Verb[commandchars=\\\{\}]}
\DefineVerbatimEnvironment{Highlighting}{Verbatim}{commandchars=\\\{\}}
% Add ',fontsize=\small' for more characters per line
\usepackage{framed}
\definecolor{shadecolor}{RGB}{248,248,248}
\newenvironment{Shaded}{\begin{snugshade}}{\end{snugshade}}
\newcommand{\AlertTok}[1]{\textcolor[rgb]{0.94,0.16,0.16}{#1}}
\newcommand{\AnnotationTok}[1]{\textcolor[rgb]{0.56,0.35,0.01}{\textbf{\textit{#1}}}}
\newcommand{\AttributeTok}[1]{\textcolor[rgb]{0.77,0.63,0.00}{#1}}
\newcommand{\BaseNTok}[1]{\textcolor[rgb]{0.00,0.00,0.81}{#1}}
\newcommand{\BuiltInTok}[1]{#1}
\newcommand{\CharTok}[1]{\textcolor[rgb]{0.31,0.60,0.02}{#1}}
\newcommand{\CommentTok}[1]{\textcolor[rgb]{0.56,0.35,0.01}{\textit{#1}}}
\newcommand{\CommentVarTok}[1]{\textcolor[rgb]{0.56,0.35,0.01}{\textbf{\textit{#1}}}}
\newcommand{\ConstantTok}[1]{\textcolor[rgb]{0.00,0.00,0.00}{#1}}
\newcommand{\ControlFlowTok}[1]{\textcolor[rgb]{0.13,0.29,0.53}{\textbf{#1}}}
\newcommand{\DataTypeTok}[1]{\textcolor[rgb]{0.13,0.29,0.53}{#1}}
\newcommand{\DecValTok}[1]{\textcolor[rgb]{0.00,0.00,0.81}{#1}}
\newcommand{\DocumentationTok}[1]{\textcolor[rgb]{0.56,0.35,0.01}{\textbf{\textit{#1}}}}
\newcommand{\ErrorTok}[1]{\textcolor[rgb]{0.64,0.00,0.00}{\textbf{#1}}}
\newcommand{\ExtensionTok}[1]{#1}
\newcommand{\FloatTok}[1]{\textcolor[rgb]{0.00,0.00,0.81}{#1}}
\newcommand{\FunctionTok}[1]{\textcolor[rgb]{0.00,0.00,0.00}{#1}}
\newcommand{\ImportTok}[1]{#1}
\newcommand{\InformationTok}[1]{\textcolor[rgb]{0.56,0.35,0.01}{\textbf{\textit{#1}}}}
\newcommand{\KeywordTok}[1]{\textcolor[rgb]{0.13,0.29,0.53}{\textbf{#1}}}
\newcommand{\NormalTok}[1]{#1}
\newcommand{\OperatorTok}[1]{\textcolor[rgb]{0.81,0.36,0.00}{\textbf{#1}}}
\newcommand{\OtherTok}[1]{\textcolor[rgb]{0.56,0.35,0.01}{#1}}
\newcommand{\PreprocessorTok}[1]{\textcolor[rgb]{0.56,0.35,0.01}{\textit{#1}}}
\newcommand{\RegionMarkerTok}[1]{#1}
\newcommand{\SpecialCharTok}[1]{\textcolor[rgb]{0.00,0.00,0.00}{#1}}
\newcommand{\SpecialStringTok}[1]{\textcolor[rgb]{0.31,0.60,0.02}{#1}}
\newcommand{\StringTok}[1]{\textcolor[rgb]{0.31,0.60,0.02}{#1}}
\newcommand{\VariableTok}[1]{\textcolor[rgb]{0.00,0.00,0.00}{#1}}
\newcommand{\VerbatimStringTok}[1]{\textcolor[rgb]{0.31,0.60,0.02}{#1}}
\newcommand{\WarningTok}[1]{\textcolor[rgb]{0.56,0.35,0.01}{\textbf{\textit{#1}}}}
\usepackage{longtable,booktabs,array}
\usepackage{calc} % for calculating minipage widths
% Correct order of tables after \paragraph or \subparagraph
\usepackage{etoolbox}
\makeatletter
\patchcmd\longtable{\par}{\if@noskipsec\mbox{}\fi\par}{}{}
\makeatother
% Allow footnotes in longtable head/foot
\IfFileExists{footnotehyper.sty}{\usepackage{footnotehyper}}{\usepackage{footnote}}
\makesavenoteenv{longtable}
\usepackage{graphicx}
\makeatletter
\def\maxwidth{\ifdim\Gin@nat@width>\linewidth\linewidth\else\Gin@nat@width\fi}
\def\maxheight{\ifdim\Gin@nat@height>\textheight\textheight\else\Gin@nat@height\fi}
\makeatother
% Scale images if necessary, so that they will not overflow the page
% margins by default, and it is still possible to overwrite the defaults
% using explicit options in \includegraphics[width, height, ...]{}
\setkeys{Gin}{width=\maxwidth,height=\maxheight,keepaspectratio}
% Set default figure placement to htbp
\makeatletter
\def\fps@figure{htbp}
\makeatother
\setlength{\emergencystretch}{3em} % prevent overfull lines
\providecommand{\tightlist}{%
  \setlength{\itemsep}{0pt}\setlength{\parskip}{0pt}}
\setcounter{secnumdepth}{5}
\newlength{\cslhangindent}
\setlength{\cslhangindent}{1.5em}
\newlength{\csllabelwidth}
\setlength{\csllabelwidth}{3em}
\newlength{\cslentryspacingunit} % times entry-spacing
\setlength{\cslentryspacingunit}{\parskip}
\newenvironment{CSLReferences}[2] % #1 hanging-ident, #2 entry spacing
 {% don't indent paragraphs
  \setlength{\parindent}{0pt}
  % turn on hanging indent if param 1 is 1
  \ifodd #1
  \let\oldpar\par
  \def\par{\hangindent=\cslhangindent\oldpar}
  \fi
  % set entry spacing
    \setlength{\parskip}{#2\cslentryspacingunit+5pt}
   }%
 {}
\usepackage{calc}
\newcommand{\CSLBlock}[1]{#1\hfill\break}
\newcommand{\CSLLeftMargin}[1]{\parbox[t]{\csllabelwidth}{#1}}
\newcommand{\CSLRightInline}[1]{\parbox[t]{\linewidth - \csllabelwidth}{#1}\break}
\newcommand{\CSLIndent}[1]{\hspace{\cslhangindent}#1}
\ifLuaTeX
\usepackage[bidi=basic]{babel}
\else
\usepackage[bidi=default]{babel}
\fi
\babelprovide[main,import]{english}
% get rid of language-specific shorthands (see #6817):
\let\LanguageShortHands\languageshorthands
\def\languageshorthands#1{}
\usepackage{booktabs}
\renewcommand{\textfraction}{0.05}
\renewcommand{\topfraction}{0.8}
\renewcommand{\bottomfraction}{0.8}
\renewcommand{\floatpagefraction}{0.75}
\usepackage[automark,headsepline=0.9pt]{scrlayer-scrpage}
%\usepackage[authordate,backend=biber]{biblatex-chicago}
\newenvironment{abstract}{}{}
\usepackage{abstract}
\usepackage{suffix}
\usepackage{xstring}
\usepackage{relsize}
\usepackage{acronym}
\usepackage{xcolor}
\usepackage{float}
\usepackage{booktabs}
\usepackage{longtable}
\usepackage{array}
\usepackage{multirow}
\usepackage{wrapfig}
\usepackage{float}
\usepackage{colortbl}
\usepackage{pdflscape}
\usepackage{tabu}
\usepackage{threeparttable}
\usepackage{threeparttablex}
\usepackage[normalem]{ulem}
\usepackage{makecell}
\usepackage{xcolor}
\ifLuaTeX
  \usepackage{selnolig}  % disable illegal ligatures
\fi

\usepackage{csquotes}
\usepackage{tocloft}

\IfFileExists{bookmark.sty}{\usepackage{bookmark}}{\usepackage{hyperref}}
\IfFileExists{xurl.sty}{\usepackage{xurl}}{} % add URL line breaks if available
\urlstyle{same} % disable monospaced font for URLs
\hypersetup{
  pdftitle={Digital Tool Box -- Project Report},
  pdfauthor={Tania Testperson (XXXX); Max Mustermann (XXXX)},
  pdflang={en},
  hidelinks,
  pdfcreator={LaTeX via pandoc}}

\title{Digital Tool Box -- Project Report}
\usepackage{etoolbox}
\makeatletter
\providecommand{\subtitle}[1]{% add subtitle to \maketitle
  \apptocmd{\@title}{\par {\large #1 \par}}{}{}
}
\makeatother
\subtitle{Project work as part of the Bachelor of Science (B.Sc.) in\\
Media Management \& Communication and Business Informatics}
\author{Tania Testperson (XXXX) \and Max Mustermann (XXXX)}
\date{}


\begin{document}

%%
%% begin titlepage
%%
\begin{titlepage}
\newgeometry{top=5cm, right=2.5cm, bottom=3cm, left=2.5cm}
\newcommand{\colorRule}[3][black]{\textcolor[HTML]{#1}{\rule{#2}{#3}}}
\begin{flushleft}
\noindent
\\[-1em]
\color[HTML]{000000}
%\makebox[0pt][l]{\colorRule[435488]{1.3\textwidth}{4pt}}
%\par
\noindent

{
	\setstretch{1.4}
	\vfill
		\noindent
	\begin{center}
		\includegraphics[width=0.6\linewidth]{./01-images/HSBA-Logo-quadratic-version-1.jpg}
	\end{center}
	\vskip 1.5cm
		\begin{center}
		{\huge \textbf{\textsf{Digital Tool Box -- Project Report}}}
	\end{center}
		\vskip 1em
	\begin{center}
		{\Large \textsf{Project work as part of the Bachelor of Science (B.Sc.) in\\
Media Management \& Communication and Business Informatics}}
	\end{center}
		\vskip 2em
	\begin{center}
		{\Large \textsf{Tania Testperson (XXXX)\\Max Mustermann (XXXX)}}
	\end{center}
	\vskip 2em
		\noindent
	\setstretch{1.05}
	\begin{center}
		{\normalsize \textsf{
				\begin{tabular}{ll}
					%% Here you can add any other information you want,
%% as long as it consists of key-value pairs that
%% are separated by an ampersand & and the line ends 
%% with a double backslash: \\

Project Theme & abcde fghij klmno pqrst uvwxyz \\
Study Group & xx \\
Lecturer & ABCDEFG \\
Submitted & \today \\ % erase the '\today' for a hardcoded date like dd.mm.yyyy \\
Team Number & yy \\


										Wordcount & 10131
\end{tabular}

							}}
	\end{center}
		\vfill
}

\textsf{}
\end{flushleft}
\end{titlepage}
\restoregeometry
\pagenumbering{arabic}

%%
%% end titlepage
%%


\pagenumbering{roman}



	\phantomsection
	\addcontentsline{toc}{chapter}{\abstractname}
	\begin{abstract}
	    \doublespacing
		The present specialized bookdown project template you are using now is intended as the basis for a report or thesis at the Hamburg School of Business Administration (HSBA). Compared to the original project template coming with the bookdown package, it generates PDF output only, using a slightly customized LaTeX template based on the original Pandoc-template, which is adapted to the requirements of the HSBA for theses and reports. The main notable differences are an appropriate title page, an automatic word count function, a list of acronyms, an adapted page numbering, the use of the appropriate Chicago Manual of Style citation style (17th edition, CSL-based) and an additional chapter with a basic declaration of honor at the end. In addition, several additional folders are used to demonstrate a basic project structure, as well as a slightly modified file organization to show the use of other R packages in addition to the bookdown basics. The contents of the original bookdown template can be found in the following chapters, the additional code examples in the appendix. For more information about this template, please refer to the file `README.md'.
	\end{abstract}


{
\setcounter{tocdepth}{3}

\phantomsection
\addcontentsline{toc}{chapter}{\contentsname}

\tableofcontents
\newpage
}


\renewcommand{\cftfignumwidth}{6em}
\renewcommand{\cftfigpresnum}{Fig. }
\phantomsection
\addcontentsline{toc}{chapter}{\listfigurename}
\listoffigures
\renewcommand{\cfttabnumwidth}{6em}
\renewcommand{\cfttabpresnum}{Tbl. }
\phantomsection
\addcontentsline{toc}{chapter}{\listtablename}
\listoftables
\newpage

\chapter*{Acronyms}
\addcontentsline{toc}{chapter}{Acronyms}
\begin{acronym}[XXXXX]
% Comment out the examples and define your own acronyms here following this pattern:
%
% \acro{code}[Short Name]{Long Name}
%
% The acronyms must be sorted by hand, they appear in the list of acronyms in the
% order the are listed here.
%
% In the text, then use \ac{code} everywhere in the text where you want to use
% the acronym / abbreviation. When used for the first time, the long name followed
% by the acronym in parenthesis will be inserted, after that the acronym only.
%
% Another example (from the documentation of LaTeX package 'acronym'):
%
% \acro{H2O}[$\mathrm{H_2O}$]{water}
%
% Then \acs{H2O} gets “H2O” and \acl{H2O} prints “water”.
%
% You can use \acp{code} in the text to use an automatic plural (s is added to
% automatically the short or long name). If the plural is not standard, use an
% additional line for the acronym definition, where the code is the same but
% short and long name is adjusted to the plural form:
%
% \acrodefplural{code}[Short Names]{Long Names}
%
% Other example commands that can be used in the text:
%
% \Ac{code}  % Works in the same way as \ac, but starts the long form with an
%            % upper case
%
% \acf{code} % If later in the text again the Full Name of the acronym should be
%            % printed, use the command
%
% \acp{code} % Works in the same way as \ac, but makes the short and/or long forms
%            % into plurals.
%
% \Acp{code} % Works in the same way as \acp, but starts the long form with an
%            % upper case letter.
%
% See the manual for a complete description:
%
% https://ctan.org/pkg/acronym?lang=en
%
\acro{cran}[CRAN]{Comprehensive R Archive Network}
% This example does not work while using the flextable package. If you use this
% package for displaying tables, then make sure you do not use the math environment
% inside of abbreviations:
\acro{H2O}[$\mathrm{H_2O}$]{water}
\acro{ide}[IDE]{integrated development environment}
\acrodefplural{ide}[IDEs]{integrated development environments}
\acro{glm}[GLM]{generalized linear model}

\end{acronym}
\clearpage

\setstretch{1.5}

\pagenumbering{arabic}
\setcounter{page}{1}

\doublespacing

\hypertarget{if-you-dont-need-the-abstract-just-comment-it-out-here-completely.-if-you-want-to}{%
\chapter{If you don't need the abstract, just comment it out here completely. If you want to}\label{if-you-dont-need-the-abstract-just-comment-it-out-here-completely.-if-you-want-to}}

Placeholder

\hypertarget{about}{%
\chapter{About}\label{about}}

Placeholder

\hypertarget{usage}{%
\section{Usage}\label{usage}}

\hypertarget{render-book}{%
\section{Render book}\label{render-book}}

\hypertarget{preview-book}{%
\section{Preview book}\label{preview-book}}

\hypertarget{hello-bookdown}{%
\chapter{Hello bookdown}\label{hello-bookdown}}

All chapters start with a first-level heading followed by your chapter title, like the line above. There should be only one first-level heading (\texttt{\#}) per .Rmd file.

\hypertarget{a-section}{%
\section{A section}\label{a-section}}

All chapter sections start with a second-level (\texttt{\#\#}) or higher heading followed by your section title, like the sections above and below here. You can have as many as you want within a chapter.

\hypertarget{an-unnumbered-section}{%
\subsection*{An unnumbered section}\label{an-unnumbered-section}}
\addcontentsline{toc}{subsection}{An unnumbered section}

Chapters and sections are numbered by default. To un-number a heading, add a \texttt{\{.unnumbered\}} or the shorter \texttt{\{-\}} at the end of the heading, like in this section.

\hypertarget{cross}{%
\chapter{Cross-references}\label{cross}}

Placeholder

\hypertarget{chapters-and-sub-chapters}{%
\section{Chapters and sub-chapters}\label{chapters-and-sub-chapters}}

\hypertarget{captioned-figures-and-tables}{%
\section{Captioned figures and tables}\label{captioned-figures-and-tables}}

\hypertarget{parts}{%
\chapter{Parts}\label{parts}}

You can add parts to organize one or more book chapters together. Parts can be inserted at the top of an .Rmd file, before the first-level chapter heading in that same file.

Add a numbered part: \texttt{\#\ (PART)\ Act\ one\ \{-\}} (followed by \texttt{\#\ A\ chapter})

Add an unnumbered part: \texttt{\#\ (PART\textbackslash{}*)\ Act\ one\ \{-\}} (followed by \texttt{\#\ A\ chapter})

Add an appendix as a special kind of un-numbered part: \texttt{\#\ (APPENDIX)\ Other\ stuff\ \{-\}} (followed by \texttt{\#\ A\ chapter}). Chapters in an appendix are prepended with letters instead of numbers.

\hypertarget{footnotes-and-citations}{%
\chapter{Footnotes and citations}\label{footnotes-and-citations}}

\hypertarget{footnotes}{%
\section{Footnotes}\label{footnotes}}

Footnotes are put inside the square brackets after a caret \texttt{\^{}{[}{]}}. Like this one \footnote{This is a footnote.}.

\hypertarget{citations}{%
\section{Citations}\label{citations}}

Reference items in your bibliography file(s) using \texttt{@key}.

For example, we are using the \textbf{bookdown} package (\protect\hyperlink{ref-R-bookdown}{Xie 2022}) (check out the last code chunk in index.Rmd to see how this citation key was added) in this sample book, which was built on top of R Markdown and \textbf{knitr} (\protect\hyperlink{ref-xie2015}{Xie 2015}) (this citation was added manually in an external file book.bib).
Note that the \texttt{.bib} files need to be listed in the index.Rmd with the YAML \texttt{bibliography} key.

The RStudio Visual Markdown Editor can also make it easier to insert citations: \url{https://rstudio.github.io/visual-markdown-editing/\#/citations}

\hypertarget{blocks}{%
\chapter{Blocks}\label{blocks}}

Placeholder

\hypertarget{equations}{%
\section{Equations}\label{equations}}

\hypertarget{theorems-and-proofs}{%
\section{Theorems and proofs}\label{theorems-and-proofs}}

\hypertarget{callout-blocks}{%
\section{Callout blocks}\label{callout-blocks}}

\hypertarget{sharing-your-book}{%
\chapter{Sharing your book}\label{sharing-your-book}}

Placeholder

\hypertarget{publishing}{%
\section{Publishing}\label{publishing}}

\hypertarget{pages}{%
\section{404 pages}\label{pages}}

\hypertarget{metadata-for-sharing}{%
\section{Metadata for sharing}\label{metadata-for-sharing}}

\pagebreak

\hypertarget{references}{%
\chapter*{References}\label{references}}
\addcontentsline{toc}{chapter}{References}

\hypertarget{refs}{}
\begin{CSLReferences}{1}{0}
\leavevmode\vadjust pre{\hypertarget{ref-R-rmarkdown}{}}%
Allaire, JJ, Yihui Xie, Jonathan McPherson, Javier Luraschi, Kevin Ushey, Aron Atkins, Hadley Wickham, Joe Cheng, Winston Chang, and Richard Iannone. 2022. \emph{Rmarkdown: Dynamic Documents for r}. \url{https://CRAN.R-project.org/package=rmarkdown}.

\leavevmode\vadjust pre{\hypertarget{ref-R-base}{}}%
R Core Team. 2022. \emph{R: A Language and Environment for Statistical Computing}. Vienna, Austria: R Foundation for Statistical Computing. \url{https://www.R-project.org/}.

\leavevmode\vadjust pre{\hypertarget{ref-xie2015}{}}%
Xie, Yihui. 2015. \emph{Dynamic Documents with {R} and Knitr}. 2nd ed. Boca Raton, Florida: Chapman; Hall/CRC. \url{http://yihui.org/knitr/}.

\leavevmode\vadjust pre{\hypertarget{ref-R-bookdown}{}}%
---------. 2022. \emph{Bookdown: Authoring Books and Technical Documents with r Markdown}. \url{https://CRAN.R-project.org/package=bookdown}.

\end{CSLReferences}

\newpage

\hypertarget{declaration-of-honor}{%
\chapter*{Declaration of Honor}\label{declaration-of-honor}}
\addcontentsline{toc}{chapter}{Declaration of Honor}

We hereby declare that

\begin{enumerate}
\def\labelenumi{\arabic{enumi}.}
\tightlist
\item
  we wrote this project report without the assistance of others;
\item
  we have marked direct quotes used from the literature and the use of ideas of other authors at the corresponding locations in the thesis;
\item
  we have not presented this thesis for any other exam. We acknowledge that a false declaration will have legal consequences.
\end{enumerate}

Hamburg, dd.mm.yyyy

aaaaa, bbbbbb

\vspace{2.5cm}

We accept that the HSBA may check the originality of our work using a range of manual and computer based techniques, including transferring and storing our submission in a database for the purpose of data-matching to help detect plagiarism.

Hamburg, dd.mm.yyyy

aaaaa, bbbbbb

\hypertarget{appendix-appendix}{%
\appendix}


Placeholder

\hypertarget{use-of-acronyms}{%
\section{Use of acronyms}\label{use-of-acronyms}}

\hypertarget{read-data}{%
\section{Load packages and read data into R}\label{read-data}}

\hypertarget{displaying-different-types-of-tables-from-modelsummary}{%
\section{Displaying different types of tables from modelsummary}\label{displaying-different-types-of-tables-from-modelsummary}}

\hypertarget{reporting-statistical-models}{%
\section{Reporting statistical models}\label{reporting-statistical-models}}

\hypertarget{t-test}{%
\subsection{\texorpdfstring{\(t\)-test}{t-test}}\label{t-test}}

\hypertarget{chi2-test}{%
\subsection{\texorpdfstring{\(\chi^2\)-test}{\textbackslash chi\^{}2-test}}\label{chi2-test}}

\hypertarget{reg-models}{%
\subsection{Linear regression models}\label{reg-models}}

\hypertarget{generalized-linear-regression-models}{%
\subsection{Generalized linear regression models}\label{generalized-linear-regression-models}}

\hypertarget{automated-report-generation}{%
\subsection{Automated report generation}\label{automated-report-generation}}

\hypertarget{plotting-statistical-models}{%
\section{Plotting statistical models}\label{plotting-statistical-models}}

\hypertarget{data-set}{%
\chapter{Data-set}\label{data-set}}

\hypertarget{boxplot-wie-viele-stuxe4dte-liegen-uxfcberunter-dem-durchschnittlichen-cli}{%
\section{Boxplot wie viele Städte liegen über/unter dem Durchschnittlichen CLI}\label{boxplot-wie-viele-stuxe4dte-liegen-uxfcberunter-dem-durchschnittlichen-cli}}

\begin{Shaded}
\begin{Highlighting}[]
\FunctionTok{library}\NormalTok{(tidyverse) }\CommentTok{\# This includes readr!}
\CommentTok{\#\textgreater{} {-}{-} Attaching packages {-}{-}{-}{-}{-}{-}{-}{-}{-}{-}{-}{-}{-}{-}{-}{-}{-}{-}{-} tidyverse 1.3.2 {-}{-}}
\CommentTok{\#\textgreater{} v ggplot2 3.3.6      v purrr   0.3.5 }
\CommentTok{\#\textgreater{} v tibble  3.1.8      v dplyr   1.0.10}
\CommentTok{\#\textgreater{} v tidyr   1.2.1      v stringr 1.4.1 }
\CommentTok{\#\textgreater{} v readr   2.1.3      v forcats 0.5.2 }
\CommentTok{\#\textgreater{} {-}{-} Conflicts {-}{-}{-}{-}{-}{-}{-}{-}{-}{-}{-}{-}{-}{-}{-}{-}{-}{-}{-}{-}{-}{-} tidyverse\_conflicts() {-}{-}}
\CommentTok{\#\textgreater{} x dplyr::filter() masks stats::filter()}
\CommentTok{\#\textgreater{} x dplyr::lag()    masks stats::lag()}
\FunctionTok{library}\NormalTok{(xtable) }\CommentTok{\# For displaying LaTeX tables}
\FunctionTok{library}\NormalTok{(modelsummary) }\CommentTok{\# For displaying regression models in tables}
\FunctionTok{library}\NormalTok{(stargazer) }\CommentTok{\# For displaying regression models in tables}
\CommentTok{\#\textgreater{} }
\CommentTok{\#\textgreater{} Please cite as: }
\CommentTok{\#\textgreater{} }
\CommentTok{\#\textgreater{}  Hlavac, Marek (2022). stargazer: Well{-}Formatted Regression and Summary Statistics Tables.}
\CommentTok{\#\textgreater{}  R package version 5.2.3. https://CRAN.R{-}project.org/package=stargazer}
\FunctionTok{library}\NormalTok{(jtools) }\CommentTok{\# For displaying regression models in tables}
\FunctionTok{library}\NormalTok{(kableExtra) }\CommentTok{\# For displaying or changing tables}
\CommentTok{\#\textgreater{} }
\CommentTok{\#\textgreater{} Attache Paket: \textquotesingle{}kableExtra\textquotesingle{}}
\CommentTok{\#\textgreater{} }
\CommentTok{\#\textgreater{} Das folgende Objekt ist maskiert \textquotesingle{}package:dplyr\textquotesingle{}:}
\CommentTok{\#\textgreater{} }
\CommentTok{\#\textgreater{}     group\_rows}
\FunctionTok{library}\NormalTok{(gt) }\CommentTok{\# For displaying tables}
\CommentTok{\#\textgreater{} }
\CommentTok{\#\textgreater{} Attache Paket: \textquotesingle{}gt\textquotesingle{}}
\CommentTok{\#\textgreater{} }
\CommentTok{\#\textgreater{} Das folgende Objekt ist maskiert \textquotesingle{}package:modelsummary\textquotesingle{}:}
\CommentTok{\#\textgreater{} }
\CommentTok{\#\textgreater{}     escape\_latex}
\FunctionTok{library}\NormalTok{(gtsummary) }\CommentTok{\# For model reporting inline and in tables}
\FunctionTok{library}\NormalTok{(broom) }\CommentTok{\# For working with statistical models}
\FunctionTok{library}\NormalTok{(car) }\CommentTok{\# For type{-}III anova tests}
\CommentTok{\#\textgreater{} Lade nötiges Paket: carData}
\CommentTok{\#\textgreater{} }
\CommentTok{\#\textgreater{} Attache Paket: \textquotesingle{}car\textquotesingle{}}
\CommentTok{\#\textgreater{} }
\CommentTok{\#\textgreater{} Das folgende Objekt ist maskiert \textquotesingle{}package:dplyr\textquotesingle{}:}
\CommentTok{\#\textgreater{} }
\CommentTok{\#\textgreater{}     recode}
\CommentTok{\#\textgreater{} }
\CommentTok{\#\textgreater{} Das folgende Objekt ist maskiert \textquotesingle{}package:purrr\textquotesingle{}:}
\CommentTok{\#\textgreater{} }
\CommentTok{\#\textgreater{}     some}
\FunctionTok{library}\NormalTok{(report) }\CommentTok{\# For automated text{-}based model reporting}
\FunctionTok{library}\NormalTok{(effects) }\CommentTok{\# For working with statistical models / visualize effects}
\CommentTok{\#\textgreater{} lattice theme set by effectsTheme()}
\CommentTok{\#\textgreater{} See ?effectsTheme for details.}
\FunctionTok{library}\NormalTok{(ggeffects) }\CommentTok{\# For working with statistical models / visualize effects}
\FunctionTok{library}\NormalTok{(patchwork) }\CommentTok{\# For putting different visualizations in one figure}
\end{Highlighting}
\end{Shaded}

\begin{Shaded}
\begin{Highlighting}[]
\NormalTok{dataset }\OtherTok{\textless{}{-}} \FunctionTok{read\_csv}\NormalTok{(}\StringTok{"02{-}data/cost{-}of{-}living{-}2017.csv"}\NormalTok{, }\AttributeTok{lazy=} \ConstantTok{FALSE}\NormalTok{)}
\CommentTok{\#\textgreater{} Rows: 511 Columns: 1}
\CommentTok{\#\textgreater{} {-}{-} Column specification {-}{-}{-}{-}{-}{-}{-}{-}{-}{-}{-}{-}{-}{-}{-}{-}{-}{-}{-}{-}{-}{-}{-}{-}{-}{-}{-}{-}{-}{-}{-}{-}{-}{-}{-}{-}}
\CommentTok{\#\textgreater{} Delimiter: ","}
\CommentTok{\#\textgreater{} chr (1): City    State   Country Cost of Living Plus Rent Ind...}
\CommentTok{\#\textgreater{} }
\CommentTok{\#\textgreater{} i Use \textasciigrave{}spec()\textasciigrave{} to retrieve the full column specification for this data.}
\CommentTok{\#\textgreater{} i Specify the column types or set \textasciigrave{}show\_col\_types = FALSE\textasciigrave{} to quiet this message.}
\end{Highlighting}
\end{Shaded}




\doublespacing

\end{document}
